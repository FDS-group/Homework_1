\documentclass{article}

\usepackage{lmodern}
\usepackage[T1]{fontenc}
\usepackage[english]{babel}
\usepackage{mathtools}
\usepackage{tocbibind}
\usepackage{multirow}
\usepackage{appendix}
\usepackage{cite}
\usepackage{bm}
\usepackage[utf8]{inputenc}
\usepackage{graphicx}
\usepackage{enumerate}
\usepackage{amsmath,amsfonts,latexsym,color,textcomp,anysize}

\begin{document}

\marginsize{3cm}{2cm}{3cm}{3cm}
\setlength{\parindent}{1cm}

\title{LPM-Effect}

\author{Bachelor Thesis 2017}
\date{Juan Mata Naranjo}

\maketitle
\tableofcontents

\newpage

\section{Introduction:}

En etse apartado buscamos plantera nuestro problema principal, que como ya sabemos es que una partícula rodeada de un determinado material tiene un bremsstrahlung diferente al que preicen Bethe-Heitler en ausencia de él. Descibrir algunos de los conceptos fundamentales como la formation length y estas cosas. En resumen, la primera parte del artículo.

\section{Bethe-Heitler Bremsstrahlung Theory:}

To begin with we will start by discussing the problem of electromagnetic radiation for charged particles in the presence of a heavier and fixed particle. As we can guess this is the simplest case of our LPM effect, which we will get to in the next section. 

These charged particles acelerate due to the presence of these other particles, emitting a radiation named Bremsstrahlung. This radiation will be the fundamental study throughout all our thesis. It is also important to figure out that this radiation emission is only important at relativistic speeds, non relativistic particles can neglect this kind of process. 

\subsection{Low Frequency Limit:}

To begin with we will consider that our incoming particle (charged particle) has low frecuencies, and therefore a low energy, that way we can use a classical approach to the problem. Having also in mind that the amplitud in non relativistic radiation depends on the product between charge and aceleration, we can safely neglect the radiation effect that our atomic particle produces. 

The radiation emitted during the colission time by a particle of charge ze is expressed as\footnote{To follow the demonstration of this formula please check Apendix 1}:

%\begin{equation}
%\label{lowfrequencyradiation}
%\dfrac{d^{2}I}{d\omega d\Omega}=\dfrac{z^{2}e^{2}}{16\pi^{3}\epsilon_{0}c}\bigg\vert \bigg\int \dfrac{d}{dt} \bigg[\dfrac{\textbf{n}\times(\textbf{n}\times\beta)}{1-\textbf{n}\beta} \bigg] e^{i\omega(t-\textbf{n}\textbf{r(t)}/c)}dt \bigg\vert^{2}
%\end{equation}

\newpage

\section*{Apendix 1: Deduction of lowfrequencyradiation}

It is not easily demonstrated that the electric field of an accelerated charged paricle has the following form:

\begin{equation}
\label{Electricfieldacceleratedcharge}
\textbf{E}(\textbf{x},t)=\dfrac{e}{4\pi\epsilon_{0}} \bigg[\dfrac{\textbf{n}-\beta}{\gamma^{2}(1-\beta\textbf{n})^{3}R^{2}}\bigg]_{ret}+\dfrac{e}{4\pi\epsilon_{0}c}\bigg[\dfrac{\textbf{n}\times{(\textbf{n}-\beta)\times d\beta/dt}}{(1-\beta\textbf{n})^{3}R}\bigg]_{ret}
\end{equation} 

\end{document}

where the sublabel 'ret' implies that we are evaluating our electric field at $t=t'+\R(t')/c$.

If we now observe this charged particle from a reference frame where it's speed is very small in comparison with the speed of light we can reduce (\ref{Electricfieldacceleratedcharge}) to:

\begin{equation}
\label{Electricfieldreduced}
\textbf{E}(\textbf{x},t)=\dfrac{e}{4\pi\epsilon_{0}c}\bigg[\dfrac{\textbf{n}\times(\textbf{n}-\beta\times d\beta/dt)}{(1-\textbf{n}\beta)^{3}R}\bigg]_{ret}
\end{equation} 

because $1/R^{2}\approx\beta\approx 0$. Using the definition of magnetic field and the Poynting vector in function of the elecrtic field we can obtain the following expression for the power radiated per solid angle:

\begin{equation}
\label{Powerradiated}
\textbf{B}=\dfrac{1}{c}[\textbf{n}\times\textbf{E}]_{ret} \ \ \ \ \ \ \ \ \ \ \ \  S=\dfrac{1}{c\mu_{0}}\vert\textbf{E}\vert^{2}\textbf{n}
\end{equation}


$$\downarrow$$


\begin{equation}
\label{Powerradiated2}
\dfrac{dP}{d\Omega}=\dfrac{1}{c\mu_{0}}\vert R\textbf{E}\vert^{2}=\vert A(t)\vert^{2}
\end{equation}

Therefore our total radiated energy can simply be obtained by integrating over all time our squared amplitud. Using the Fourier tranformations so that we can express this energy as a function of our frecuencie we get:\footnote{We have used:


$$\dfrac{dW}{d\Omega}=\int^{\infty}_{-\infty}\vert\textbf{A}(t)\vert^{2}dt$$
 
$$\textbf{A}(\omega)=\dfrac{1}{2\pi}\int^{\infty}_{-\infty}\textbf{A(t)}e^{i\omega t}dt$$ 

$$\textbf{A}(t)=\dfrac{1}{2\pi}\int^{\infty}_{-\infty}\textbf{A}(\omega)e^{-i\omega t}dt$$
}


\begin{equation}
\label{Energíapersolidangle}
\dfrac{dW}{d\Omega}=\int^{\infty}_{-\infty}d\omega\int^{\infty}_{-\infty}d\omega'\textbf{A}^{*}(\omega')\textbf{A}(\omega)\int^{\infty}_{-\infty}\dfrac{dt}{2\pi}e^{i(\omega'-\omega)t
\end{equation}

Realising that the last part of our integral is a delta function $\delta(\omega'-\omega)$ we finally obtain:

\begin{equation}
\label{Powerradiateddef}
\dfrac{dW}{d\Omega}=\int^{\infty}_{-\infty}\vert \textbf{A}(w)\vert^{2}d\omega = \int^{\infty}_{0}\vert \textbf{A}(w)\vert^{2}d\omega
\end{equation}

The fact that we can express the energy per solid angle as a function of $\omega$ and t equaly is a direct consecuence of the Parseval theorem. We have also neglected the negative frecuencies since they are physically  not relevant. Defining the total energy radiated per solid angle as the integral over all the $\omega$ as the radiated energy and having in mind that since $\textbf{A}(t)$ is real it must be true that $\textbf{A}(-w)=\textbf{A}^{*}(\omega)$ we get to the following expresion:

\begin{equation}
\label{Energyradiated}
\dfrac{d^{2}I}{d\omegad\Omega}=2\vert\textbf{A}(\omega)\vert^{2}
\end{equation}  

Using the expresion of $\textbf{A}(t)$, (\ref{Powerradiated2}) and (\ref{Electricfieldreduced}) we get to the following term:

\begin{equation}
\label{Amplitudfrecuencia}
\textbf{A}(\omega)=\bigg (\dfrac{e^{2}}{32\pi^{3}\epsilon_{0}c}\bigg )^{1/2} \bigg\int^{\infty}_{-\infty}e^{i\omega t}\bigg[\dfrac{\textbf{n}\times[(\textbf{n}-\beta)\timesd\beta/dt]}{(1-\beta\textbf{n})^{3}}\bigg]_{ret}dt
\end{equation}

Doing a change of  variable we can pass from the retarded time to our proper time. For that its important remember $t=t'+R(t')/c$ and to realise that $dt'/dt=(1-\textbf{n}\beta)$. We can also approximate $R(t')\approx x-\textbf{n}\textbf{r}(t')$. So with all this and (\ref{Energyradiated}) we finally reach:

\begin{equation}
\label{Energyradiatedalmost}
\dfrac{d^{2}I}{d\omega d\Omega}=\dfrac{z^{2}e^{2}}{16\pi^{3}\epsilon_{0}c}\bigg\vert \bigg\int \dfrac{\textbf{n}\times[(\textbf{n}-\beta)\times d\beta/dt]}{(1-\beta\textbf{n})^{2}} e^{i\omega(t-\textbf{n}\textbf{r(t)}/c)}dt \bigg\vert^{2}
\end{equation}

If we now realise that we are located at a long distance from our particle we can deduce that the normal vector \textbf{n} will practically stay constant ($dn/dt\approx 0$), so we can replace:

\begin{equation}
\label{igualdad1}
\dfrac{\textbf{n}\times[(\textbf{n}-\beta)\times d\beta/dt]}{(1-\beta\textbf{n})^{2}}=\dfrac{d}{dt}\bigg[\dfrac{\textbf{n}\times(\textbf{n}\times\beta}{(1-\beta\textbf{n})}\bigg]
\end{equation}

With this we have shown from where our first and probably principal equation comes from.

\end{document}